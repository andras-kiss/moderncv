%% start of file `template.tex'.
%% Copyright 2006-2015 Xavier Danaux (xdanaux@gmail.com).
%
% This work may be distributed and/or modified under the
% conditions of the LaTeX Project Public License version 1.3c,
% available at http://www.latex-project.org/lppl/.


\documentclass[11pt,a4paper,roman]{moderncv}        % possible options include font size ('10pt', '11pt' and '12pt'), paper size ('a4paper', 'letterpaper', 'a5paper', 'legalpaper', 'executivepaper' and 'landscape') and font family ('sans' and 'roman')

% moderncv themes
\moderncvstyle{classic}                             % style options are 'casual' (default), 'classic', 'banking', 'oldstyle' and 'fancy'
\moderncvcolor{black}                               % color options 'black', 'blue' (default), 'burgundy', 'green', 'grey', 'orange', 'purple' and 'red'
%\renewcommand{\familydefault}{\sfdefault}         % to set the default font; use '\sfdefault' for the default sans serif font, '\rmdefault' for the default roman one, or any tex font name
%\nopagenumbers{}                                  % uncomment to suppress automatic page numbering for CVs longer than one page

% character encoding
\usepackage[utf8]{inputenc}                       % if you are not using xelatex ou lualatex, replace by the encoding you are using
%\usepackage{CJKutf8}                              % if you need to use CJK to typeset your resume in Chinese, Japanese or Korean

% adjust the page margins
\usepackage[scale=0.75]{geometry}
%\setlength{\hintscolumnwidth}{3cm}                % if you want to change the width of the column with the dates
%\setlength{\makecvtitlenamewidth}{10cm}           % for the 'classic' style, if you want to force the width allocated to your name and avoid line breaks. be careful though, the length is normally calculated to avoid any overlap with your personal info; use this at your own typographical risks...

\usepackage{lastpage}
\rfoot{\addressfont\itshape\textcolor{gray}{Page \thepage\ of \pageref{LastPage}}}
\lfoot{\addressfont\itshape\textcolor{gray}{2016.08.31}}
%\chead{\addressfont\itshape\textcolor{gray}{Curriculum vit\ae~ -- András Kiss}}
\lhead{\addressfont\itshape\textcolor{gray}{Curriculum vit\ae}}
\rhead{\addressfont\itshape\textcolor{gray}{András Kiss}}

% personal data
\name{András}{Kiss}
\title{Curriculum vit\ae}                               % optional, remove / comment the line if not wanted
\address{Móricz Zsigmond tér 3.}{7624 Pécs}{Hungary}% optional, remove / comment the line if not wanted; the "postcode city" and "country" arguments can be omitted or provided empty
\phone[mobile]{+36~(20)~388~1324}                   % optional, remove / comment the line if not wanted; the optional "type" of the phone can be "mobile" (default), "fixed" or "fax"
\phone[fixed]{+36~(72)~501~500~61021}
\phone[fax]{+36~(72)~501~518}
\email{akiss@gamma.pte.ttk.pte.hu}                               % optional, remove / comment the line if not wanted
\homepage{http://kemia.ttk.pte.hu/fizkem}                         % optional, remove / comment the line if not wanted
%\social[linkedin]{john.doe}                        % optional, remove / comment the line if not wanted
%\social[twitter]{jdoe}                             % optional, remove / comment the line if not wanted
\social[github]{andras-kiss}                              % optional, remove / comment the line if not wanted
%\extrainfo{additional information}                 % optional, remove / comment the line if not wanted
\photo[75pt][0.4pt]{self.eps}                       % optional, remove / comment the line if not wanted; '64pt' is the height the picture must be resized to, 0.4pt is the thickness of the frame around it (put it to 0pt for no frame) and 'picture' is the name of the picture file

%\quote{Some quote}                                 % optional, remove / comment the line if not wanted

% bibliography adjustements (only useful if you make citations in your resume, or print a list of publications using BibTeX)
%   to show numerical labels in the bibliography (default is to show no labels)
\makeatletter\renewcommand*{\bibliographyitemlabel}{\@biblabel{\arabic{enumiv}}}\makeatother
%   to redefine the bibliography heading string ("Publications")
%\renewcommand{\refname}{Articles}

% bibliography with mutiple entries
%\usepackage{multibib}
%\newcites{book,misc}{{Books},{Others}}
%----------------------------------------------------------------------------------
%            content
%----------------------------------------------------------------------------------
\begin{document}
%\begin{CJK*}{UTF8}{gbsn}                          % to typeset your resume in Chinese using CJK
%-----       resume       ---------------------------------------------------------
\makecvtitle

\section{Current occupation}
\cvitem{position}{Assistant lecturer (2015--)}
\cvitem{institution}{Department of General and Physical Chemistry}
\cvitem{}{University of Pécs}
\cvitem{}{7622, Pécs, Ifjúság útja 6.}

\section{Education}
\cventry{1997-2003}{High school diploma}{Kisfaludy Károly Highschool}{Mohács}{Grade: \textit{4.6/5 (Excellent)}}{}
\cventry{2003--2011}{Biologist}{University of Pécs}{Pécs}{Grade: \textit{4.4/5 (Good)}}{}  % arguments 3 to 6 can be left empty
\cventry{2011--2014}{PhD (Doctoral candidate)}{University of Pécs}{Pécs}{}{}

\section{Master's thesis}
%\cvitem{original title}{\emph{Szén-dioxid mikrocella fejlesztése és alkalmazása PEKM mérőcsúcsként. Élesztőtelep szén-dioxid kibocsátásának modellszámításos becslése.}}
\cvitem{title}{\emph{Development and application of a carbon-dioxide microcell as SECM microtip. Estimation of carbon-dioxid output of yeast colonies by model calculations.}}
\cvitem{supervisor}{Dr. Nagy Géza DSc. professor emeritus}
\cvitem{defended}{2011}

\section{Doctoral dissertation}
%\cvitem{original title}{\emph{Szén-dioxid mikrocella fejlesztése és alkalmazása PEKM mérőcsúcsként. Élesztőtelep szén-dioxid kibocsátásának modellszámításos becslése.}}
\cvitem{tentative title}{\emph{Recent advances in Scanning Electrochemical Microscopy}}
\cvitem{supervisor}{Dr. Nagy Géza DSc. professor emeritus}
\cvitem{tb. defended}{2016}

\section{Languages}
\cvitemwithcomment{Hungarian}{Mother tongue}{}
\cvitemwithcomment{English}{Advanced C1}{2016.02.16. BME Advanced language exam, certificate: \emph{no.} 1309673}
\cvitemwithcomment{German}{Beginner}{}

\section{Area of focus}
\cvitem{}{Electrochemistry, microelectrodes, Scanning Electrochemical Microscopy, corrosion, analytical chemistry, numerical simulations.}

\section{Computer skills}
\cvdoubleitem{\emph{Programming}}{C, C++, Fortran, Java, Bash script}{\emph{Graphics}}{Inkscape, CorelDRAW, Gnuplot, Gimp}
\cvdoubleitem{\emph{Word processing}}{Microsoft products, \LaTeXe}{\emph{Plotting}}{Gnuplot, Tikz, Origin, Qtiplot}
\cvdoubleitem{\emph{SA}}{Linux, Windows, BSD, UNIX}{\emph{Version control}}{git, github}

\section{Scholarships, internships and short visits}
%\subsection{Vocational}
\cventry{2005.08.01--2005.08.26}{Summer internship}{Balaton Limnologial Research Institue}{Tihany}{}{Effect of various salt concentration on freshwater alg\ae.}

\cventry{2006}{Internship}{Department of Biophysics, University of Pécs}{Pécs}{}{Investigation of the interaction of actin and titin.}

\cventry{2006}{Internship}{Department of Ecology, University of Pécs}{Pécs}{}{Study on the ecology of small mammals.}

\cventry{2006.07.31--2006.08.25}{Summer internship}{Balaton Limnologial Research Institue}{Tihany}{}{Effect of turbidity and depth on the picoalg\ae composition of lake Balaton.}

\cventry{2007--2009}{Internship}{Department of Microbiology, University of Pécs}{Pécs}{}{Studying the oxidative stress induction effect of patulin on \emph{Schizosaccharomyces cerevisi\ae}\newline Studying the carcinogenic and mutagenic effect of primycin, a new antibiotics; with DEL and Ames tests.}

\cventry{2009.07--2009.09}{Research}{Masaryk University}{Brno, Czech Republic}{}{Developing a tyrosinase based polyphenol sensor. \newline Investigation of adhesion of mammalian cells on the surface of quartz microbalances.}

\cventry{2010.08.01--2010.09.05}{Research}{Masaryk University}{Brno, Czech Republic}{}{Development of a selective polyphenol sensor.}

\cventry{2009--2011}{Internship, main focus}{Department of General and Physical Chemistry, University of Pécs}{Pécs}{}{Development of a CO$_2$ microcell, SECM scanning, simulation of diffusion.}

\cventry{2012.05.17--2012.06.16}{Research}{University of La Laguna}{Canary Islands, Spain}{}{Fabrication of a low resistance Mg$^{2+}$-ion selective micropipette electrode for potentiometric Scanning Electrochemical Microscopy monitoring of microgalvanic corrosion processes.}

\cventry{2013.03.06--2013.06.26}{Erasmus scholarship}{\AA bo Akademi}{Turku, Finland}{}{Improving the lower detection limit of ion-selective microelectrodes. \newline Development of a conductivity based airborne carbon nanotube sensor.}

\cventry{2013.09--2014.08}{Apáczai Csere János Scholarship}{University of Pécs}{Pécs}{}{Investigation of corrosion processes with Scanning Electrochemical Microscope, TÉT-12-RO-1-2013-0018, TÁMOP-4.2.2.A-11/1/KONV-2012-0065}

\cventry{2015.10.22--2015.10.26}{Short visit}{Department of Analytical Chemistry, University of Regensburg}{Regensburg, Germany}{}{}

\cventry{2016.10.20--2016.02.28}{Short visit}{Department of Physical Chemistry, Ibn Zohr University}{Agadir, Morocco}{}{}

\section{Publications}
\cvitem{1.}{András Kiss, Laszló Kiss, Barna Kovács, Géza Nagy, Air Gap Microcell for Scanning Electrochemical Microscopic Imaging of Carbon Dioxide Output. Model Calculation and Gas Phase SECM Measurements for Estimation of Carbon Dioxide Producing Activity of Microbial Sources, \emph{Electroanalysis 2011, 23, No. 10, 2320 – 2326}}
\cvitem{2.}{Ricardo M. Souto, András Kiss, Javier Izquierdo, Lívia Nagy, István Bitter, Géza Nagy, Spatially resolved imaging of concentration distributions on corroding magnesium-based materials exposed to aqueous environments by SECM, \emph{Electrochemistry Communications, Volume 26, January 2013, Pages 25-28}}
\cvitem{3.}{András Kiss, Ricardo M. Souto, Géza Nagy, Investigation of Mg/Al alloy sacrificial anode corrosion with Scanning Electrochemical Microscopy, \emph{Periodica Polytechnica Chemical Engineering, 57/1–2 (2013) 11–14. doi: 10.3311/PPch.2164}}
\cvitem{4.}{A. Kiss, J. Izquierdo, J.J. Santana, L. Nagy, I. Bitter, G. Nagy, and R.M. Souto, Development of Mg$^{2+}$ ion-selective microelectrodes for potentiometric Scanning Electrochemical Microscopy monitoring of galvanic corrosion processes, \emph{Journal of The Electrochemical Society 160, no. 9 (2013): C451-C459}}
\cvitem{5.}{A Kiss, G Nagy, New SECM scanning algorithms for improved potentiometric imaging of circularly symmetric targets, \emph{Electrochimica Acta 119, 169-174}}
\cvitem{6.}{Zs Őri, A Kiss, AA Ciucu, C Mihailciuc, CD Stefanescu, L Nagy, G Nagy, Sensitivity enhancement of a ,,bananatrode'' biosensor for dopamine based on SECM studies inside its reaction layer, \emph{Sensors and Actuators B: Chemical 190, 149-156}}
\cvitem{7.}{A. Kiss, G. Nagy, Deconvolution of potentiometric SECM images recorded with high scan rate, \emph{Electrochimica Acta, article in press, doi:10.1016/j.electacta.2015.02.096}}
\cvitem{8.}{A. Kiss, G. Nagy, Deconvolution in potentiometric SECM, \emph{Electroanalysis Mátrafüred special issue, article in press, DOI: 10.1002/elan.201400598}}
\cvitem{9.}{Izquierdo, J., Fernández-Pérez, B.M., Filotás, D., Őri, Z., Kiss, A., Martín-Gómez, R.T., Nagy, L., Nagy, G. and Souto, R.M., 2016. Imaging of Concentration Distributions and Hydrogen Evolution on Corroding Magnesium Exposed to Aqueous Environments Using Scanning Electrochemical Microscopy, \emph{Electroanalysis, 2016.}}


\section{Presentations}
\cvitem{1.}{CO$_2$ Partial Pressure Imaging in Gas Phase with Scanning Electrochemical Microscopy (SECM), Poster, \emph{X. CECE Conference, Pécs, 2010.}}

\cvitem{2.}{Selective Amperometric Determination Of Pyrocatechol and Phenol in Wines with Flow-Injection Analysis, Poster, \emph{X. CECE Conference, Pécs, 2010.}}

\cvitem{3.}{Four-Channel Enzyme Biosensor for Determination of Phenols in Wine, Poster, \emph{X. CECE Conference, Pécs, 2010.}}

\cvitem{4.}{Development of a CO$_2$ microcell, and its application as measuring tip in Scanning Electrochemical Microscope. Scanning in gas phase over biological samples, Presentation, \emph{XXXIV. Szegedi Kémiai Előadói Napok, Szeged, 2011.}}

\cvitem{5.}{Investigation of Mg/Al alloy sacrificial anode corrosion with Scanning Electrochemical Microscopy, Poszter, \emph{Chemical Engineering Workshop ’12, Veszprém, 2012.}}

\cvitem{6.}{Investigation of galvanic corrosion of the Fe-Mg galvanic pair with Scanning Electrochemical Microscope, Poster, \emph{Chemical Sensors Workshop ’12, Pécs, 2012.}}

\cvitem{7.}{Fabrication of a new, solid contact Mg$^{2+}$ ion-selective electrode, and its application in Scanning Electrochemical Microscopic corrosion studies, Presenttion, \emph{1st Doctoral Workshop on Natural Sciences, Pécs, 2012.}}

\cvitem{8.}{A new, solid contact Mg$^{2+}$ ion-selective electrode as measuring tip for Scanning Electrochemical Microscope in corrosion studies, Presentation, \emph{János Szentágothai Memorial Conference and Student Competition, Pécs, 2012 October 29-30.}}

\cvitem{9.}{New insights in the corrosion mechanism of magnesium by SECM, Presentation, \emph{7th Workshop on Scanning Electrochemical Microscopy (SECM) and Related Techniques, Ein Gedi, Israel, February 17-21, 2013.}}

\cvitem{10.}{High-speed potentiometric SECM imaging of radially symmetric targets, Presentation, \emph{ESEAC Malmö, Sweden, 11-14 June 2013.}}

\cvitem{11.}{Deconvolution of potentiometric SECM images recorded with high scanrate, Poster, \emph{Mátrafüred Konferencia 2014 Június 13-16, Visegrád, Hungary.}}


%\clearpage
%-----       letter       ---------------------------------------------------------
% recipient data
%\recipient{Company Recruitment team}{Company, Inc.\\123 somestreet\\some city}
%\date{January 01, 1984}
%\opening{Dear Sir or Madam,}
%\closing{Yours faithfully,}
%\enclosure[Attached]{curriculum vit\ae{}}          % use an optional argument to use a string other than "Enclosure", or redefine \enclname
%\makelettertitle

%Lorem ipsum dolor sit amet, consectetur adipiscing elit. Duis ullamcorper neque sit amet lectus facilisis sed luctus nisl iaculis. Vivamus at neque arcu, sed tempor quam. Curabitur pharetra tincidunt tincidunt. Morbi volutpat feugiat mauris, quis tempor neque vehicula volutpat. Duis tristique justo vel massa fermentum accumsan. Mauris ante elit, feugiat vestibulum tempor eget, eleifend ac ipsum. Donec scelerisque lobortis ipsum eu vestibulum. Pellentesque vel massa at felis accumsan rhoncus.

%Suspendisse commodo, massa eu congue tincidunt, elit mauris pellentesque orci, cursus tempor odio nisl euismod augue. Aliquam adipiscing nibh ut odio sodales et pulvinar tortor laoreet. Mauris a accumsan ligula. Class aptent taciti sociosqu ad litora torquent per conubia nostra, per inceptos himenaeos. Suspendisse vulputate sem vehicula ipsum varius nec tempus dui dapibus. Phasellus et est urna, ut auctor erat. Sed tincidunt odio id odio aliquam mattis. Donec sapien nulla, feugiat eget adipiscing sit amet, lacinia ut dolor. Phasellus tincidunt, leo a fringilla consectetur, felis diam aliquam urna, vitae aliquet lectus orci nec velit. Vivamus dapibus varius blandit.

%Duis sit amet magna ante, at sodales diam. Aenean consectetur porta risus et sagittis. Ut interdum, enim varius pellentesque tincidunt, magna libero sodales tortor, ut fermentum nunc metus a ante. Vivamus odio leo, tincidunt eu luctus ut, sollicitudin sit amet metus. Nunc sed orci lectus. Ut sodales magna sed velit volutpat sit amet pulvinar diam venenatis.

%Albert Einstein discovered that $e=mc^2$ in 1905.

%\[ e=\lim_{n \to \infty} \left(1+\frac{1}{n}\right)^n \]

%\makeletterclosing

%\clearpage\end{CJK*}                              % if you are typesetting your resume in Chinese using CJK; the \clearpage is required for fancyhdr to work correctly with CJK, though it kills the page numbering by making \lastpage undefined
\end{document}


%% end of file `template.tex'.
